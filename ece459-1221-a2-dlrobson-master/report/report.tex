\documentclass[12pt]{article}

\usepackage[letterpaper, hmargin=0.75in, vmargin=0.75in]{geometry}
\usepackage{float}

\pagestyle{empty}

\title{ECE 459: Programming for Performance\\Assignment 2}
\author{Daniel Robson}
\date{\today}

\begin{document}

\maketitle

\section{Message Passing}
To begin, the alphabet letters are evenly split among the number of cpus on the system.
After that, an unbounded channel is created, which will hold the found secret.
These senders and receivers are cloned and passed to the threads.
Once a solution has been found, the secret will be sent to the shared channel.
Each resulting thread checks the channel each iteration, and will stop and halt when a message exists in the channel.

\section{Shared Memory}
The shared memory solution is similar to the message passing solution.
Several aspects were able to be reused.
This solution also begins by dividing the letters between the threads.
Instead of a channel, an atomic bool is created and set to false.
The threads are spun up, each containing a mutable reference to a clone of the atomic bool.
The check\_all function is modified to take a mutable reference of the bool.
When this bool is set to true, that indicates that a secret is found.
Since each iteration this bool is checked, the function cancels and returns.


\section{Valgrind Errors}
There are none!

\section{Helgrind Errors}
There are some Helgrind errors, however they are linked to the imported libraries.


\section{Benchmark Results}
When we run the command to get an output of saaaa. Below are the benchmarks:

Original implementation: 26.75s

Message Passing: 1.76s

Shared Memory: 0.01s

Given a different command where the expected key is 01234:

Original implementation: 40.26s

Message Passing: 3.54s

Shared Memory: 2.26s

Overall, it's much quicker with parallelization!

\end{document}

